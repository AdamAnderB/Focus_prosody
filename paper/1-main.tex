% This template is based on the apa7 latex class.
% I recommend reading about the class here:
% https://ctan.math.illinois.edu/macros/latex/contrib/apa7/apa7.pdf
% The class includes various useful commands which all format documents in the correct apa7 style.

% Documentlass can be stu (student), doc (document), jou (journal), or man (manuscript)
% Use stu for Course papers
\documentclass[man, floatsintext, letterpaper, 12pt, donotrepeattitle]{apa7}
\usepackage[style=apa, backend=biber]{biblatex}
\addbibresource{my_bib.bib}
\usepackage[american]{babel}


%\usepackage{fontspec}        % For specifying fonts
%\setmainfont{Times New Roman} % Font for the main text
\usepackage{newtxtext} 


%%%%% PACKAGES

\usepackage{tipa}
\usepackage{csquotes}
\usepackage{subfiles}
%this is off because I figured out a process for making it a word file but it requires that the numbers are off to do it
%\usepackage[right]{lineno}%numbers on the right
%\linenumbers
\usepackage{lipsum}
\usepackage{xcolor}
%hyperlink colors fix
\usepackage{listings}
\usepackage{soul}


%ATTENTION ATTENTION ATTENTION ATTENTION ATTENTION 
%without yellow 

%\sethlcolor{white}
%with yellow
\sethlcolor{yellow}
%ATTENTION ATTENTION ATTENTION ATTENTION ATTENTION 

\definecolor{mygreen}{rgb}{0,0.6,0}
\definecolor{mygray}{rgb}{0.5,0.5,0.5}
\definecolor{mymauve}{rgb}{0.58,0,0.82}
\definecolor{orange}{RGB}{253,181,21}
\definecolor{custombg}{RGB}{220,220,220} % Adjust the background color
\definecolor{pink}{RGB}{255,105,180}
\definecolor{customgreen}{RGB}{0,128,0}
\definecolor{customorange}{RGB}{255,165,0}
\definecolor{bluethread}{RGB}{8,146,208}
\definecolor{codegray}{RGB}{246,246,246}
\definecolor{cmu_red}{RGB}{196,18,48}
\definecolor{rose_red}{RGB}{239,58,71}
\definecolor{cmu_blue}{RGB}{4,54,115}
\definecolor{green}{RGB}{0,150,71}
\definecolor{cmu_grey}{RGB}{109,110,113}
%\definecolor{gorilla_red}{RGB}{231,18,52}
\definecolor{gorilla_red}{RGB}{254,243,245}

\definecolor{yellow}{RGB}{255,255,0}


\usepackage{booktabs}
\usepackage{amssymb}
\hypersetup{
    colorlinks,
    linkcolor={blue!50!black},
    citecolor={blue!50!black},
    urlcolor={blue!80!black}
}

\newcommand{\inlineR}[1]{{\footnotesize\texttt{\colorbox{codegray}{#1}}}}

\newcommand{\gorilla}[1]{{\small\texttt{\colorbox{gorilla_red}{\textcolor{black}{\fontfamily{ptm}\selectfont \textit{#1}}}}}}


\usepackage{datatool}
\usepackage{ifthen}

% Load the CSV file
%\DTLloaddb{chunknums}{scripts/chunk-num_lister.csv}

% Define a command to look up content
%\newcommand{\liner}[1]{%
%  \DTLforeach{chunk_nums}{\col1=content_cleaned, \col2=Line}{%
%    \ifthenelse{\equal{#1}{\col1}}{\col2}{}%
%  }%
%}

%\newcommand{\getValue}[1]{%
%  \DTLforeach{chunk-num}{\ColumnOne=content_cleaned,\ColumnTwo=Line}{%
%    \ifthenelse{\equal{\ColumnOne}{#1}}{%
%      \edef\storedValue{\ColumnTwo}%
%    }{%
%      \typeout{No match for #1 in \ColumnOne}%
%    }%
%  }%
%}

% Define the command to get the value

%\usepackage{titlesec}

%\titleformat{\section}
%  {\normalfont\Roman\Big\bfseries\color{black}}
%  {\thesection}
%  {1em}
%  {\thesection\hspace{0.6em}}
%\titleformat{\subsection}
%  {\normalfont\Roman\Big\it\color{black}}
%  {\thesubsection}
%  {1em}
%  {}
%\titleformat{\subsubsection}
%  {\normalfont\Roman\Big\it\color{black}}
%  {\thesubsubsection}
%  {1em}
%  {}
\title{} % Needed for \maketitle to work
\shorttitle{FiREE Replication Framework}
\author{} % Leave empty if desired
\affiliation{} % Leave empty to avoid formatting issues

% Suppress the visual display of the title block
\title{} % Needed for \maketitle to compile
\shorttitle{FiREE Replication Framework} 
\author{}
\affiliation{}

% Suppress the visual output of \maketitle
\makeatletter
\renewcommand{\@maketitle}{\relax}
\makeatother

\setlength{\parindent}{30pt}
\setlength{\parskip}{0pt plus 1pt}

% Title
\begin{document}

% Your custom title page
\begin{titlepage}
\centering
\vspace*{1cm}

{\LARGE \bfseries Focus (on) Replication: Focus Processing in L1 and L2 English using the Fidelity, Refinement, and Exploratory Extension (FiREE) Replication Framework \par}

\vspace{1.5cm}

{\large Adam A. Bramlett\textsuperscript{*} and Seth Wiener\textsuperscript{*,†} \par}

\vspace{1em}

\makebox[0.9\textwidth][c]{%
\parbox{0.9\textwidth}{%
\centering
\textsuperscript{*}Department of Languages, Cultures, and Applied Linguistics,\\
Carnegie Mellon University, 341 Posner Hall, Pittsburgh, PA 15213, USA
}}

\vspace{1em}

{\small †Corresponding author: \href{mailto:sethw1@cmu.edu}{sethw1@cmu.edu} \par}

\vspace{1.5cm}

\section*{Declarations of Interests}
\noindent The authors declare no financial support or conflicts of interest. Experimental materials, unidentifiable data, and code are openly available and shared through the Open Science Framework at: \url{https://osf.io/wa4gv/?view_only=de113dbced6b46fab96ca8217b3c1ca6}

\vspace{1cm}

\section*{Acknowledgments}
\noindent The authors would like to thank the authors of \cite{ge2021a} for both sharing their materials and their original contribution to our understanding of L1 and L2 focus processing.

\end{titlepage}


% Now your normal content
\subfile{paper/2-abstract}
\subfile{paper/3-introduction}
\subfile{paper/4-methods}
\subfile{paper/5-results}
%\subfile{paper/5-extension_results}
\subfile{paper/6-discussion}
\subfile{paper/7-conclusion}
%\subfile{outline}

% sections are done with \section, \subsection and \subsubsection.
% for more information, see https://ctan.math.illinois.edu/macros/latex/contrib/apa7/apa7.pdf

\printbibliography
% include all sources in references.bib
% cite sources in text with \autocite, or:
% \parencite for "(Author, 2023)"
% \textcite for "Author (2023)"
\end{document}
