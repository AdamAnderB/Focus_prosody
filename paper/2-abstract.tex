A fundamental challenge in replication studies lies in balancing methodological fidelity with statistical conservatism while also leaving space for deeper theoretical exploration. This paper introduces a replication framework designed to balance these sometimes competing interests: the Fidelity, Refinement, and Exploratory Extension (FiREE) Replication Framework. We demonstrate this framework by attempting to replicate an in person eye-tracking experiment \citep{Ge2021} using entirely web-based methods. Fidelity ensures adherence to the original design for direct comparison, while Refinement enhances analytical rigor through improved statistical modeling, including Generalized Additive Models (GAMs) to track fixation dynamics over time. Exploratory Extension investigates individual differences in cognitive and perceptual abilities to assess their role in focus processing across individual stimuli acoustics. Our replication did not confirm the original study’s claim that native speakers integrate prosodic cues more efficiently. Instead, Dutch speakers showed earlier competitor fixations, challenging prior assumptions. Refinement analyses confirmed the absence of a robust L1-L2 difference, while Exploratory Extension revealed that focus processing was primarily driven by individual differences in acoustic sensitivity, auditory-motor integration, and the acoustics of the utterance itself. These findings underscore the complexity of decision making for doing replications as well as the inherent variability of capturing prosodic processing effects, highlighting the need to consider methodological rigor alongside individual variability in future studies.
