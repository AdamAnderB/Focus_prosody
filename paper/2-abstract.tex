\section{Abstract}
A fundamental challenge in replication studies lies in balancing methodological fidelity with statistical conservatism while also leaving space for deeper theoretical exploration. This paper introduces a  replication framework designed to balance these sometimes competing interests: the Fidelity, Refinement, and Exploratory Extension (FiREE) Replication Framework. We demonstrate this framework by using entirely web-based methods \hl{to conduct a close replication of an} in-person English focus processing eye-tracking experiment \parencite{ge2021a}, which found L1 Dutch-L2 English learners showed delayed fixation patterns to focus alternatives compared to L1 English speakers \parencite{mcmanus2024replication2}. We selected this study due to the unexpected L1 Dutch effect, the study's recency and open materials, and its compatibility with individual difference extensions. We adhered closely to the fidelity of the original methodology and analysis. We recruited L1 English and L1 Dutch-L2 English participants via the internet and partially reproduced previous findings by capturing efficient L1 processing and less efficient L2 processing. We then refined the analytical approach and treated time as a continuous, dynamic variable using Generalized Additive Models. Differences between L1 and L2 participants were found not to be static processing deficits, but rather dynamic changes in fixation patterns over time. We show that the L1 speaker advantage may be one of long-term consistency rather than early efficiency. In our exploratory extension, we further tested whether eye movements can be predicted using acoustic measurements of the stimuli and five individual difference measures. We found evidence of acoustics, participants' perceptual auditory sensitivity, and auditory motor reproduction abilities predicting eye fixations in line with interaction views of auditory processing. Our FiREE framework therefore ensures that replicability is not merely about reproducing a prior effect, but also about clarifying whether findings are meaningful, interpretable, and generalizable across different statistical, methodological, and theoretical contexts.

