\subsection{Fidelity, Refinement, and Exploratory Extension (FiREE) Replications}

A researcher faces many challenges to successfully replicate previously published research. Crucially, the experimental design must be identical and the same analytical methods must be applied to the new data. With the shift to web-based data collection, researchers have moved in person experiments to online platforms such as Gorilla \citep{Anwyl-Irvine_2019} and crowd-sourced data from sites such as Prolific \citep{douglas2023data}. Current web-based platforms are astonishingly effective \citep{tomczak2023over, eerola2021online}. Researchers are now able to replicate even in person eye-tracking results using web-based methods and data collection \citep[see ][] {semmelmann2018online, Vos_2022, Prystauka_Altmann_Rothman_2023}. With faster computers, open source software like R \citep{mizumoto_r_2015, Wickham2017R}, and advances in analytic methods \citep{gries2017ten}, researchers now have a powerful analytic toolbox at their disposal. 

Yet, attempting to replicate an in person study with web-based methods is not so straightforward. This paper is concerned with getting these design and analysis details right. Critical details, such as experimental scripts, pre-processing steps, or precise modeling decisions need to be made available to researchers for an accurate and precise replication design. Historically, this lack of transparency was rarely, if ever, intentional; journal space constraints, limitations in reporting conventions, the need for concise academic writing, the lack of the Internet, and so on. Without access to original scripts or comprehensive documentation, even minor procedural variations can introduce discrepancies that complicate replication. \cite{AOW}, for example, illustrates this challenge: if details such as a visual analog slider's starting position or exact data processing steps are not explicitly provided, attempts to replicate the study risk diverging from the original implementation in ways that may meaningfully affect the results. Addressing these challenges requires a replication framework that ensures fidelity to the original study, while also allowing for methodological refinement and theoretical extension.

To this end, the present study introduces the Fidelity, Refinement, and Exploratory Extension, or "FiREE" Replication framework. FiREE integrates confirmatory replication, methodological refinement, and theory-driven exploratory analysis. The fidelity component ensures the adherence to the original study design, using the same procedures and statistical models to assess whether the original findings hold under comparable conditions. Maintaining methodological consistency facilitates direct comparisons and contributes to broader efforts to establish replicability across fields. 

The refinement component strengthens the analytical framework by incorporating improved statistical practices to address issues such as multiple comparisons, overfitting, and model complexity. This step improves the statistical rigor and ensures that any observed effects were not artifacts of less conservative analytical approaches. Moreover, this step ensures that effects were not overlooked due to less robust modeling approaches. Replications must not only confirm previous findings but also evaluate whether the findings are robust to improved methodologies. We know that the same construct can be studied with diverse methodologies, which can yield different results \citep[e.g.,][] {roettger2017methodological}. That is, distinguishing between effects that are method-dependent and those that hold across multiple different analytical choices is crucial.

Finally, exploratory extensions allow for the systematic examination of theoretical assumptions that may not have been explicitly tested in the original study. Whereas exploratory extensions are not strictly hypothesis-driven, this component is guided by theoretical considerations that extend beyond the immediate replication framework. Rather than engaging in post hoc data exploration, exploratory extensions provide a structured, theory-motivated examination of factors, such as individual differences, which may contribute to the results. This approach is particularly valuable when prior studies have assumed a particular explanatory framework, such as working memory as the primary contributor to an effect, without systematically testing alternative explanations.

By adopting the FiREE Replication approach, researchers can move beyond a binary success-or-failure replication framework \citep{Nosek_Errington2020} and paint a fuller picture of their replications. Rather than treating replication as a rigid exercise in repeating previous studies, we conceptualize it as an opportunity to evaluate methodological robustness, refine statistical practices, and uncover theoretical insights that were not explicitly addressed in the original research. This perspective frames replication as an active tool for scientific progress, reinforcing that replicability is not merely about reproducing a prior effect, but also about ensuring that findings are meaningful, interpretable, and generalizable across different statistical, methodological, and theoretical contexts.

In what follows, we first discuss the linguistic concept of focus and the findings of \cite{Ge2021}---the focus processing eye-tracking study we set out to partially replicate. We then briefly review five areas in which individuals differ in their language processing behavior. These areas serve as predictors in the exploratory extension. All our methods, materials, code, and data are freely available on the Open Science Framework. Using our \cite{Ge2021} replication results, we show and tell the reader how to balance methodological fidelity with statistical rigor. We suggest potential refinements where alternative analyzes could be performed. Finally, we connect our replication and extension findings to larger theories of psycholinguistics. 

\subsection{Focus in English \textit{only}-sentences}

Information in languages must be organized and presented in a manner that clearly and effectively conveys the intention of the speaker. The study of how information is structured in language is wide and encompasses syntax, semantics, pragmatics, and prosody \citep[see ][] {Breen2010, Lambrecht1994, Roberts2012}. An important area of information structure is focus or the information that is considered most important, relevant, new or contrastive \citep{Kiss1998}. Focus is believed to be a linguistic universal \citep{Comrie1989}. Yet, how focus is implemented varies across languages and can be constrained by a language’s phonology and morphosyntax (\citep{Kiss1998, Lambrecht1994}. How speakers process focus in sentences remains a rich area of psycholinguistics research as it can reveal much about language and cognition. The methods in which focus has been researched range from behavioral tasks \citep[e.g.,][] {Cutler1979, Paterson1999} to event-related-potentials \citep[e.g.,][] {Chen2014, Wang2011} to eye-tracking \citep[e.g.,][] {Filik2005, Hohle2016}.

In the present replication study, we “focus” on English preverbal \textit{only}-sentences with varying positions of prosodic prominence. As an example, take the sentence, “Obama only vetoed the bill.” The scope of the focus particle \textit{only} can be associated with the verb ‘vetoed’ or the object ‘the bill.’ Semantic parsing, however, depends on which word(s) carries prosodic prominence, which is generally conveyed through an expanded F0 range, increased amplitude, and longer duration (i.e., nuclear pitch accent on the focal element(s); \citep{Breen2010, Gussenhoven1983}.  If ‘the bill’ carries prosodic prominence, the listener will understand that Obama vetoed nothing else but the bill. In contrast, if ‘vetoed’ carries prosodic prominence, the listener will understand that Obama did nothing else to the bill other than veto it. 

Processing focus in \textit{only}-sentences requires multiple levels of linguistic knowledge and serves as a valuable test case to understand how the processing of the first (L1) and second language (L2) differs. \cite{Ge2021}---the study we set out to partially replicate---examined how L1 and L2 English speakers process \textit{only}-sentences in real time. The authors made use of the look and listen visual world paradigm in which a participant looks at images on screen while listening to spoken sentences. Importantly, the images on the screen represented the intended target of the focus or the alternative focus (i.e., a competitor). For example, each experimental sentence stimulus contained \textit{only} with prosodic prominence on either the verb or the object, creating two conditions as in, "The dinosaur is only CARRYING the bucket, not throwing the bucket" (verb condition) or, "The dinosaur is only carrying the BUCKET, not carrying the suitcase" (object condition).

\cite{Ge2021} tested L1 English speakers and L2 English learners whose L1 was either Cantonese or Dutch. Dutch, like English, uses prosodic prominence to realize focus through an expanded F0 range, increased amplitude, and longer durations \citep{dimitrova2010focus}. Dutch \textit{only}-sentences (or \textit{alleen}-sentences, the Dutch equivalent) can pattern like English \textit{only}-sentences as in "De dinosaurus draagt alleen De EMMER" (The dinosaur is only carrying the BUCKET). Importantly, Dutch \texit{alleen}-sentences can also place \textit{only} after the object as in "De dinosaurus DRAAGT De emmer alleen" (The dinosaur is only CARRYING the bucket). In contrast, Cantonese \textit{only}-sentences are considerably different from those in English. Cantonese has a number of different focus particles, which makes prosody somewhat optional for realizing focus \citep{lee2019focus, wu2010prosodic, ge2024bilingual, fung2000final}. 

\cite{Ge2021} found that L1 Dutch-L2 English and L1 Cantonese-L2 English speakers showed patterns of eye movements that differed from those of L1 English speakers. L1 English speakers considered the alternative of focus at an early stage (generally before hearing "not" in sentences). L2 speakers showed delayed eye movements to the alternative of focus (generally while or after hearing "not" in sentences) with the L1 Dutch speakers showing even more delayed behavior than the L1 Cantonese speakers. The authors interpreted these differences as evidence for problematic integration of multiple interfaces (e.g., syntax-semantics, syntax-pragmatics) in real time. The Prosodic-Learning Interference Hypothesis \citep{tremblay2016effects, tremblay2021re}, states that L2 learning of prosodic cues is more difficult when the L1 and L2 use similar prosodic cues as in Dutch-English and less difficult when the L1 and L2 use different prosodic cues as in Cantonese-English, which also uses spoken particles as cues.

Taken together, \cite{Ge2021}'s in-person eye-tracking study demonstrates L2 speakers showed delayed eye movements relative to L1 speakers when considering the focus alternative. Additionally, the prosodic cues involved in the L1 can affect acquisition of L2 prosodic cues (see also \cite{ge2021comprehension}). L1 Dutch speakers showed even more delayed eye movements relative to L1 Cantonese speakers presumably due to the similarity between the way English and Dutch realize focus. We set out to partially replicate \cite{Ge2021} using web-based eye-tracking and open materials and code. We collected L1 English and L1 Dutch data but not L1 Cantonese data given current geopolitical constraints. We also extend \cite{Ge2021} by probing multiple individual differences to determine what, if any, behavioral measures serve as reliable predictors of focus processing in an L1 and L2. 

\subsection{Individuals differ in their language processing behavior}
We know that individuals differ in nearly every linguistic measure. There are a finite number of things that researchers can measure, and these measurements have informed our understanding of the mechanisms involved in language acquisition and processing \cite{Kidd2018, skehan1991individual, nelson1981individual}. 

Current experimental evidence suggests that there is a complex interplay between cognitive abilities, auditory perceptual abilities, and motor reproduction abilities during speech processing \citep{saito2022does, bramlett_wiener_24_speechprosody, bakkouche2025effects, Kachlicka_Saito_Tierney_2019}. Whereas we cannot review (and explore) every possible individual difference predictor, we briefly review the following five ways in which individuals differ as it relates to English language processing: working memory, cognitive control, English lexical knowledge, auditory perception abilities, and auditory motor reproduction abilities.


\subsubsection{Working memory}
Working memory, or the ability to hold recent input in mind and later draw on it (see \cite{baddeley2003working,carpenter2013role}) can affect a wide range of linguistic processes. How working memory affects the processing of focus prosody is unclear. To the best of our knowledge, this specific research question has not been asked. Based on findings related to working memory and prosody processing research (e.g., \cite{traxler2009hierarchical,ferreira2015prosody, bishop2021exploring}), the answer most likely depends on the population, task demands, and methods used. For example, working memory, as measured by the digit span task (the same task we use in the present study), was not predictive in terms of identifying English emotional prosody \citep{sinagra2022perception} or Italian lexical stress \citep{ppcc} in groups of L1 adults, presumably given the relatively simple task demands of identifying happy/sad prosody or a penultimate/antepenultimate stressed word. In the present study, however, we use longer sentences involving potential object or verb targets that may require listeners to store linguistic information and draw on it later (i.e., upon hearing the "not" in the sentence a listener must recall the earlier input). We also test both L1 and L2 listeners. It is not clear whether the task demands will place a burden on the listener, hence the exploratory nature for this variable. 


\subsubsection{Cognitive control}

\subsubsection{English Lexical proficiency}
L1 and L2 speakers differ in their lexical proficiency, which has unsurprisingly been shown to contribute to differences in language processing \citep{Yap2012, zareva2005relationship}. Here, we are interested in whether increased English lexical proficiency, as measured by the LexTALE task \citep{lemhofer2012introducing}, leads to earlier looks for the L2 English participants. That is, does a greater proficiency result in more "L1-like" eye movements? Although this particular research question has not been asked about focus processing (that we are aware of), there is evidence that higher proficiency scores are correlated with the speed of activation of the target and the degree to which lexical competition is resolved \citep{sarrett2022within}. It is not clear whether the "look-and-listen" nature of our task will yield similar results, thus our exploratory question.

 

\subsubsection{Perceptual Auditory sensitivity}
Detecting prosodic cues requires a certain level of sensitivity to incoming acoustics. English and Dutch both use F0 range (or what listeners perceive as pitch) expansion, amplitude changes (intensity rise-time), and duration increases to convey focus. A growing body of research has explored auditory processing in L2 speech learning. Here we look at four measures of what \cite{saito2023does} calls "explicit acuity in L2 speech learning:" how sensitive a listener is to temporal and spectral cues or dimensions (e.g., formant,
pitch, duration, and intensity). These measures have proven to be reliable measures for a range of L2 speech learning tasks \citep{Kachlicka_Saito_Tierney_2019, bakkouche2025effects, bramlett_wiener_24_speechprosody}. Here we extend this work to test whether these abilities are predictive of focus processing in \textit{only}-sentences. Once again, we are unaware of any previous research that has attempted to use these measures to predict L1 or L2 focus processing.


\subsubsection{Auditory Motor reproduction}
pitch
duration



\subsubsection{Predictions guiding our exploratory analysis} 
We test whether participants with greater working memory look to the alternative focus object at an earlier point in time than those with reduced working memory. If there is an effect of working memory, we assume it will be found among all participants and not just within the L1 or L2 group. 



CC - higher CC, easier to turn off L1. Effect only for L2

Lex - higher lex, more L1-like behavior. Effect only for L2

Perception - 
Motor - 