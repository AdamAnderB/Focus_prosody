
\section{Results}
\subsection{Replication}
\subsubsection{Partial Replication: Testing prior findings}

The foundation of any replication study is fidelity—ensuring that the methodological and analytical choices align as closely as possible with the original study. In this section, our primary goal is to faithfully replicate the methodological choices of \textcite{ge2021a}, ensuring that our analyses align with the original study’s design, statistical procedures, and reporting conventions. This includes using the same data transformations and inferential techniques to ensure that our replication remains as close as possible to the original study. Any unavoidable deviations, whether due to differences in implementation, data collection platform, or sample composition, are explicitly documented to clarify the degree to which methodological fidelity has been maintained.

Following \textcite{ge2021a}, we analyzed fixation proportions to target objects, object-stressed competitors, verb-stressed competitors, and distractors across time, comparing conditions in which the object or verb was stressed. To maintain fidelity, we used a comparable time-binning approach, aligned with the beginning and end of each word in the sentence (e.g., “The rabbit is”, “only”, “verb1”, “the1”, “object1”, “gap”, “not”, “verb2”, “the2”, “object2”, “offset”). This follows \textcite{ge2021a}’s critical word boundaries and allows for direct comparisons between studies with nine critical time bins. We calculated mean fixation proportions and standard errors for each interest area across time bins and conditions. Like \textcite{ge2021a}, we plotted these fixation patterns using a time-series approach, with separate visualizations for each competitor type and experimental group. L1 English fixation proportions across areas of interest (AOI) for both our study and and estimated result of \textcite{ge2021a} can be found in Figure \ref{fig:english_fix}; L1 Dutch-L2 English fixation proportions can be found in Figure \ref{fig:dutch_fix}.

\begin{figure}[H]  % 'p' puts it on its own page
    \centering
    \includegraphics[width=\textwidth,height=\textheight,keepaspectratio]{viz/english_fix.png}
    \caption{L1 English speakers' fixation proportions across time bins from \textcite{ge2021a} (top row) and our study (bottom row) by target, competitors, and distractor.}
    \label{fig:english_fix}
\end{figure}

\begin{figure}  % 'p' puts it on its own page
    \centering
    \includegraphics[width=\textwidth,height=\textheight,keepaspectratio]{viz/dutch_fix.png}
    \caption{L1 Dutch-L2 English speakers' fixation proportions across time bins from \textcite{ge2021a} (top row) and our study (bottom row) by target, competitors, and distractor.}
    \label{fig:dutch_fix}
\end{figure}

Following \textcite{ge2021a}, we applied mixed-effects regression models to predict fixation proportions as a function of focus condition, time, and AOI for each time bin separately. We fit a series of linear mixed-effects models (LMMs) with random intercepts for participants and slopes for condition and AOI. The best-fitting models were selected based on lowest AIC. Following \textcite{ge2021a}, we used the baseline of object stressed competitor for condition and a baseline of object stressed competitor for AOI. This means that negative effects for either variable can be interpreted as object facing and positive effects are verb facing. Said another way, a positive effect for AOI means more looks toward verb competitor. Likewise, a positive effect for condition indicates more looks during verb-focused sentences.

Unlike \textcite{ge2021a}, our model selection pipeline incorporated an automated model comparison approach, iteratively testing models with simpler effects structures. This change was necessary because many models did not converge. The full model included both AOI and condition, along with their interaction, as well as random intercepts for participants and slopes for items. If this failed to converge or created a singularity, then the model was reduced to remove individual slopes but still included the AOI × condition interaction. If that model failed to converge, the AOI x condition interaction was removed, leaving only the main effects. The model was then further simplified by removing participant-specific effects entirely, keeping the overall effects of AOI and condition and participant random slopes. If none of these models converged or if the simplest model was the best, then the last step was a basic linear model without any participant-level adjustments, treating all data points as independent. This yielded 18 statistical model comparisons (9 critical time bins x 2 L1 groups). We present results in order from the beginning of the sentence to the end. 

Starting with the L1 English models, three significant effects were found: A positive effect was found for AOI ($\beta$ = 0.062, \textit{SE} = 0.029, \textit{t} = 2.17, \textit{p} = 0.032), indicating more looks to verb competitors than object competitors during the “not” time bin. A positive interaction between AOI and condition was found ($\beta$ = 0.175, \textit{SE} = 0.056, \textit{t} = 3.14, \textit{p} = 0.0019) for the “offset” time bin, indicating that participants began to look more at the verb competitor during the verb-focused sentences. Additionally, in the same “offset” time bin, a negative effect was found for condition ($\beta$ = -0.091, \textit{SE} = 0.039, \textit{t} = -2.30, \textit{p} = 0.022) indicating more looks to competitors during object-focused sentences overall during the offset time bin. Our L1 English participants' results can be seen in comparison to \textcite{ge2021a} in Figure \ref{fig:model_plot_english}.

\begin{figure}[H]  % 'p' puts it on its own page
    \centering
    \includegraphics[width=\textwidth,height=\textheight,keepaspectratio]{viz/model_plot_english.png}
    \caption{Model outputs for L1 English participants across nine modeled time bins. Order of time bins appears in order across the top. Estimates of \textcite{ge2021a} data appears on top while our data appears below. Significance levels 0.05, 0.01, and 0.001 are indicated by *, **, and ***, respectively above the estimate. Purple and yellow references AOI, like that of figures \ref{fig:english_fix2} and \ref{fig:dutch_fix2}. Blue and orange reference condition, like that of figures \ref{fig:english_fix} and \ref{fig:dutch_fix}.}
    \label{fig:model_plot_english}
\end{figure}

For the L1 Dutch-L2 English models, nine significant effects were found: A negative effect of AOI ($\beta$ = -0.099, \textit{SE} = 0.041, \textit{t} = -2.43, \textit{p} = 0.017) was found for “the” time bin, indicating more looks to object competitor AOIs. Similarly, a negative effect for AOI was found for “object1” ($\beta$ = -0.117, \textit{SE} = 0.033, \textit{t} = -3.59, \textit{p} $<$ 0.001), “the2” ($\beta$ = -0.129, \textit{SE} = 0.042, \textit{t} = -3.08, \textit{p} = 0.0025), and “object2” ($\beta$ = -0.091, \textit{SE} = 0.032, \textit{t} = -2.83, \textit{p} = 0.0055) time bins, all of which indicate more looks to the object competitors AOIs. Further, a negative effect of condition was found for both “object2” ($\beta$ = 0.180, \textit{SE} = 0.045, \textit{t} = 3.98, \textit{p} $<$ 0.001) and “offset” ($\beta$ = -0.130, \textit{SE} = 0.050, \textit{t} = -2.58, \textit{p} = 0.011), indicating more looks to competitors during verb-focused sentences. Lastly, a positive interaction between AOI and condition was found during the “object2” ($\beta$ = 0.180, \textit{SE} = 0.045, \textit{t} = 3.98, \textit{p} $<$ 0.001) and “offset” ($\beta$ = 0.226, \textit{SE} = 0.071, \textit{t} = 3.17, \textit{p} = 0.0019) time bins, indicating more looks to object competitors during object-focused sentences. Our L1 Dutch-L2 English participant results can be seen in comparison to \textcite{ge2021a}'s results in Figure \ref{fig:model_plot_english}.


\begin{figure}[H]  % 'p' puts it on its own page
    \centering
    \includegraphics[width=\textwidth,height=\textheight,keepaspectratio]{viz/model_plot_dutch.png}
    \caption{Model outputs for L1 Dutch-L2 English participants across nine modeled time bins. Order of time bins appears in order across the top. Estimates of \textcite{ge2021a} data appears on top while our data appears below. Significance levels 0.05, 0.01, and 0.001 are indicated by *, **, and ***, respectively above the estimate. Like Figure \ref{fig:model_plot_english}, purple and yellow reference AOI, like that of figures \ref{fig:english_fix2} and \ref{fig:dutch_fix2}. Blue and orange reference condition, like that of figures \ref{fig:english_fix} and \ref{fig:dutch_fix}.}
    \label{fig:model_plot_dutch}
\end{figure}





\subsubsection{Methodological refinement: A “focused” approach}

Whereas fidelity in replication is essential, it is also necessary to evaluate whether the original analytical choices align with current best practices. The refinement phase of our analysis addresses potential limitations such as model specification, overfitting, multiple comparisons, and analytical transparency. Here, we implement refined statistical approaches that maintain the interpretability of the original findings while increasing statistical robustness. By comparing our refined results to both the original and replicated findings, we can assess whether methodological improvements impact the observed effects and whether the key conclusions of \textcite{ge2021a} remain stable across different analytical approaches.

The first refinement we make is analyzing both binary competitor fixations and target fixations separately, rather than relying on aggregated measures. This approach provides a more comprehensive view of participants’ behavior, capturing differences in how they allocate visual attention. Secondly, we combine the analysis from both language groups to be able to compare both within language by condition and across participant L1. Fig \ref{fig:english_fix2} shows a new way of plotting Figure \ref{fig:english_fix}, which allows the reader to compare when fixations to specific AOI deviate from each other and not just across conditions.

\begin{figure}[H]  % 'p' puts it on its own page
    \centering
    \includegraphics[width=\textwidth,height=\textheight,keepaspectratio]{viz/english_fix2.png}
    \caption{Fixation proportions for L1 English participants across object stressed (left) and verb stressed (right) sentences. Estimates of \textcite{ge2021a}'s data appears on top while our data appears below.}
    \label{fig:english_fix2}
\end{figure}

For these refined models, we present both target models and competitor models. That is, we compare the target fixations (green lines in Figures \ref{fig:english_fix2} and \ref{fig:dutch_fix2}) in the target model and we are comparing the object competitor fixations (yellow lines in Figures \ref{fig:english_fix2} and \ref{fig:dutch_fix2}) in the object-focused sentences (left plots of Figures \ref{fig:english_fix2} and \ref{fig:dutch_fix2}) as well as the verb competitor fixations (purple lines in Figures \ref{fig:english_fix2} and \ref{fig:dutch_fix2}) during verb-focused sentences (right plots of Figures \ref{fig:english_fix2} and \ref{fig:dutch_fix2}) for the competitor models. 

\begin{figure}[H]  % 'p' puts it on its own page
    \centering
    \includegraphics[width=\textwidth,height=\textheight,keepaspectratio]{viz/dutch_fix2.png}
    \caption{Fixation proportions for L1 Dutch-L2 English participants across object stressed (left) and verb stressed (right) sentences. Estimates of \cite{ge2021a}'s data appears on top while our data appears below.}
    \label{fig:dutch_fix2}
\end{figure}

An additional difference in our refined model is that we do all time bins in two models: we split the sentence at the gap between the first phrase (e.g., “The rabbit only verb1 the1 object1) and second phrase (e.g., “not verb2 the2 object2). This yields four total models (target/competitor x phrases one/two). 

Unlike the linear mixed-effects models (LMMs) used in the fidelity modeling, here we use Generalized Additive Models (GAMs) to better capture time-dependent changes in fixation proportions. Since eye-tracking data unfolds continuously over time, GAMs provide a more flexible way to model the non-linear dynamics of fixations that may not be well captured by traditional LMMs \parencite{Wood2017}. In this approach, we compare two GAM variants: a full interaction model, which includes a three-way interaction between time, experiment group, and condition, and a main effects model, which assumes that changes in fixation proportions follow an additive pattern without time-dependent interactions. Both models include random smooth effects for participants, allowing for individual variability while capturing group-level fixation trends. Max models started with random items but were removed as none of the models converged.

For the first phrase of the target models, a positive main effect of stress was found ($\beta$ = 0.155, \textit{SE} = 0.053, \textit{t} = 2.93, \textit{p} = 0.003), indicating more target fixations during object focus sentences. A positive main effect of time was found ($\beta$ = 0.067, \textit{SE} = 0.012, \textit{t} = 5.66, \textit{p} $<$ 0.001), indicating increased looks to targets over the first phrase in general. A negative two-way interaction between L1 and stress was found ($\beta$ = -0.316, \textit{SE} = 0.084, \textit{t} = -3.76, \textit{p} $<$ 0.001), indicating fewer looks during verb stressed stimuli for L1 English participants. Additionally, we found a negative two-way interaction between time and stress in phrase 1 of the target model ($\beta$ = -0.036, \textit{SE} = 0.017, \textit{t} = -2.12, \textit{p} = 0.034), indicating more verb looks to targets later on in phrase one for verb-focused sentences. Lastly, the positive three-way interaction between time, L1, and stress ($\beta$ = 0.136, \textit{SE} = 0.027, \textit{t} = 5.12, \textit{p} $<$ 0.001) indicates that over time, English participants exhibited an increasing trend in fixations to the target in verb-stressed sentences, in contrast to L1 Dutch-L2 English participants.
 
 For target fixations during the second phrase, a main effect of L1 was found ($\beta$ = 0.834, \textit{SE} = 0.237, \textit{t} = 3.52, \textit{p} $<$ 0.001), indicating that English speakers had significantly more looks to targets during the second phrase. Additionally, a negative effect of time was found ($\beta$ = -0.137, \textit{SE} = 0.012, \textit{t} = -11.36, \textit{p} $<$ .001), indicating fewer looks to targets over the second phrase. A negative two-way interaction between L1 and time was also found ($\beta$ = -0.058, \textit{SE} = 0.019, \textit{t} = -3.09, \textit{p} = 0.002), indicating more looks to targets for L1 Dutch-L2 English participants over time.

For the competitor first phrase model, a positive effect of time was found ($\beta$ = 0.056, \textit{SE} = 0.012, \textit{t} = 4.63, \textit{p} $<$ 0.001), indicating more looks to competitors over time during the first phrase. A negative two-way interaction between time and stress was found ($\beta$ = -0.095, \textit{SE} = 0.018, \textit{t} = -5.26, \textit{p} $<$ 0.001), indicating more looks to verb competitors as time increases. No significant effects were found for the second phrase competitor model. Lastly, a negative two-way interaction between L1 and stress was found ($\beta$ = -0.178, \textit{SE} = 0.086, \textit{t} = -2.06, \textit{p} = 0.040), indicating more looks to the verb competitor for L1 Dutch-L2 English participants.

For the second phrase competitor model, a positive effect of time was found ($\beta$ = 0.161, \textit{SE} = 0.014, \textit{t} = 11.42, \textit{p} $<$ 0.001), indicating that more looks to competitors occurred over the duration of the second phrase. A positive interaction between time and L1 was also found ($\beta$ = 0.055, \textit{SE} = 0.022, \textit{t} = 2.50, \textit{p} = 0.012), indicating that L1 Dutch-L2 English speakers looked to competitors more as time went on over the second phrase. 

\begin{figure}[H]  % 'p' puts it on its own page
    \centering
    \includegraphics[width=\textwidth,height=\textheight,keepaspectratio]{viz/gam_mod_out.png}
    \caption{Generalized additive model output across phrase one (left column) and phrase two (right column). Significance levels 0.05, 0.01, and 0.001 are indicated by *, **, and ***, respectively above the estimate. Like Figures \ref{fig:model_plot_english} and \ref{fig:model_plot_dutch}, blue and orange reference condition based on reference levels in the model, like that of Figures \ref{fig:english_fix} and \ref{fig:dutch_fix}, language is indicated by black (L1 Dutch-L2 English) and gray (L1 English), based on reference levels.}
    \label{fig:gam_mod_out}
\end{figure}

\subsection{Exploratory extension}

Beyond replication and refinement, we extend the analysis to explore six extensions of \cite{ge2021a}. Specifically, we ask how fine-grained acoustic-phonetic properties of the stimuli affect participants' eye movements (EE1), and whether working memory (EE2), cognitive control (EE3), English lexical proficiency (EE4), perceptual auditory sensitivity (EE5), and auditory motor reproduction (EE6) affect participants' eye movements. The individual differences among our participants are visualized in Figure \ref{fig:combined_plot}, which presents a comprehensive view of the multidimensional nature of these differences. Figure \ref{fig:acoustic_faceted} presents the acoustic results.


\begin{figure}[H]  % 'p' puts it on its own page
    \centering
    \includegraphics[width=\textwidth,height=\textheight,keepaspectratio]{viz/combined_plot_circle.png}
    \caption{In this plot of individual differences, there are three sections A, B, and C. A- shows tasks across the L1 Dutch-L2 English and L1 English participants to give a distributional view of scores. B- provides individual difference vectors for each participant (stacked bar plots). The circular stacked bar plots are centered on a labeled 0 line so that negative and positive scores are stacked away from each other. Similarly, on the outer edge of the circular stacked bar plot the individual scores for tasks are shown in concentric circles by group (melodic and rhythmic scores, cognitive measures, and acoustic sensitivity), like the bar plots, inward or outward deviations from the black lines indicates negative and positive values. C shows simple background information such as age by L1 and location tested as well as number of languages learned and participant education.}
    \label{fig:combined_plot}
\end{figure}
\clearpage


We used our refined modeling approach as a baseline procedure. To optimize model selection while maintaining interpretability, we employed LASSO-GAM Feature Selection, combining LASSO regression (cv.glmnet(); \parencite{Friedman2010} with Generalized Additive Models (GAMs) (mgcv; \parencite{Wood2017}. First, LASSO identified the most predictive variables from a set of individual differences (e.g., cognitive abilities, auditory perception, etc.), acoustic features (e.g., pitch, duration, etc.), and experimental conditions, while regularizing to prevent overfitting. Next, the selected predictors were incorporated into a GAM to model nonlinear time effects and random participant variability. This data-driven approach ensures that only the most informative factors contribute to the model, balancing flexibility, interpretability, and predictive accuracy.

The LASSO was applied in a generalized linear model (GLM) with a binomial link function, where the design matrix included individual differences, acoustic properties, and experimental conditions, along with key interaction terms. The individual difference measures included working memory capacity, drift rate (decision-making efficiency), lexical proficiency (LexTALE score), and motor reproduction. The acoustic features included pitch range (min-max pitch per word), duration, stress prominence (spectral tilt in lower frequencies), and amplitude (raw and dB-scaled). The experimental conditions included focus condition (object-focused vs. verb-focused) and participant group assignment (e.g., L1).

To account for potential interactive effects, we included key two-way and three-way interactions. These interactions tested whether perceptual, cognitive, and linguistic factors jointly influenced fixations. Specifically, we modeled interactions between individual differences and acoustic properties (e.g., pitch sensitivity × pitch range × duration) and interactions between experimental conditions and perceptual abilities (e.g., focus condition × working memory, L1 background × stress prominence). Additionally, we included a three-way interaction to examine whether the effect of working memory on lexical processing varied as a function of drift rate and lexical knowledge.

In the target model of phrase one, no main effects were found. However, a negative two-way interaction between duration d' and the duration of the word was found ($\beta$ = -0.848, \textit{SE} = 0.205, \textit{t} = -4.14, \textit{p} $<$ 0.001), indicating that fewer target looks occurred for those with lesser duration sensitivity during stimuli that have shorter durations. Similarly, a positive two-way interaction between formant d$'$ and word stress was found ($\beta$ = 0.230, \textit{SE} = 0.106, \textit{t} = 2.16, \textit{p} = 0.031), indicating more target fixations for those with better formant sensitivity for words with greater word stress. Lastly, a positive two-way interaction between pitch d$'$ and pitch range was found ($\beta$ = 0.144, \textit{SE} = 0.068, \textit{t} = 2.11, \textit{p} = 0.035), indicating that more target fixations occurred for individuals with greater pitch sensitivity for words with a larger pitch range.

 For the second phrase target model, a negative effect of duration was found ($\beta$ = -0.357, \textit{SE} = 0.093, \textit{t} = -3.85, \textit{p} = 0.0001), indicating that shorter word acoustics lead to fewer target fixations. A negative effect of stress was also found ($\beta$ = -0.060, \textit{SE} = 0.022, \textit{t} = -2.72, \textit{p} = 0.0066), indicating that verb-focused sentences generally had fewer target fixations during the second phrase. A negative two-way interaction between duration d$'$ and word duration was also found ($\beta$ = -0.396, \textit{SE} = 0.125, \textit{t} = -3.18, \textit{p} = 0.0015), indicating fewer target fixations for those participants with lesser duration sensitivity during stimuli that have shorter duration. Finally, a negative interaction between pitch range and melody reproduction was found ($\beta$ = -0.487, \textit{SE} = 0.110, \textit{t} = -4.42, \textit{p} $<$ .00001), indicating that fewer target fixations occurred during phrases with lower pitch for those with lower melodic reproduction abilities.

 For the first phrase competitor model, a negative effect of duration was found ($\beta$ = -0.477, \textit{SE} = 0.195, \textit{t} = -2.45, \textit{p} = .014), indicating fewer looks to competitors for shorter duration word bins. A positive effect of L1 was also found ($\beta$ = 0.325, \textit{SE} = 0.120, \textit{t} = 2.71, \textit{p} = .007), indicating more looks to competitors during the first syllable for L1 Dutch-L2 English participants. A positive effect of word stress was also found ($\beta$ = 0.197, \textit{SE} = 0.097, \textit{t} = 2.03, \textit{p} = .042), indicating more competitor looks for words with higher word stress. In terms of two-way interactions, a positive interaction between duration d$'$ and duration of the word was found ($\beta$ = 0.853, \textit{SE} = 0.216, \textit{t} = 3.96, \textit{p} $<$ .001), indicating more competitor looks when words have longer durations during the first phrase. The effect is mirrored in a positive interaction between formant d$'$ and word stress ($\beta$ = 0.419, \textit{SE} = 0.116, \textit{t} = 3.61, \textit{p} $<$ .001), which indicates more competitor looks for those with greater formant sensitivity when words have high word stress. A positive interaction between melody reproduction and pitch range ($\beta$ = 0.239, \textit{SE} = 0.072, \textit{t} = 3.33, \textit{p} $<$ .001) further indicates that for participants with higher melodic reproduction abilities, words with greater pitch range lead to more competitor fixations during the first phrase. Finally, a negative interaction between stress and L1 ($\beta$ = -0.267, \textit{SE} = 0.047, \textit{t} = -5.64, \textit{p} $<$ .001) indicates fewer competitor fixations for L1 English speakers during verb stressed first phrases.

Our last model is the second phrase competitor model, where a positive effect of duration was found ($\beta$ = 0.362, \textit{SE} = 0.091, \textit{t} = 3.96, \textit{p} $<$ .001), indicating more competitor fixations for words with shorter duration during the second phrase. A negative effect of pitch d$'$ was also found ($\beta$ = -0.393, \textit{SE} = 0.194, \textit{t} = -2.03, \textit{p} = .043), indicating fewer competitor fixations for participants with lower pitch sensitivity. Likewise, a positive effect of stress was found ($\beta$ = 0.355, \textit{SE} = 0.077, \textit{t} = 4.63, \textit{p} $<$ .001), indicating greater competitor fixations during the second phrase for object-focused phrases. A two-way negative interaction between L1 and stress was found ($\beta$ = -0.188, \textit{SE} = 0.052, \textit{t} = -3.64, \textit{p} $<$ 0.001), indicating fewer competitor fixations for English speakers during verb focus sentences in the second phrase. Finally, a positive interaction between pitch d$'$ and pitch range was found ($\beta$ = 0.664, \textit{SE} = 0.131, \textit{t} = 5.07, \textit{p} $<$ 0.001), indicating more competitor looks when a participant has more pitch sensitivity and the word has higher pitch range. All results for the exploratory extension analyses can be found in Figure \ref{fig:id_gam_mod_out}. Table \ref{tab:fixation_patterns} summarizes the significant findings of the exploratory study.

\begin{figure}[H]  % 'p' puts it on its own page
    \centering
    \includegraphics[width=\textwidth,height=\textheight,keepaspectratio]{viz/id_gam_mod_out.png}
    \caption{All non significant values remain white. Significant values are colorized by individual difference measure, like that of figure \ref{fig:combined_plot}. Gray indicates significant but not an individual difference measure.}
    \label{fig:id_gam_mod_out}
\end{figure}


\begin{table}[h]
    \centering
    \renewcommand{\arraystretch}{1.3}
    \resizebox{\textwidth}{!}{
    \begin{tabular}{|l|c|l|c|l|}
        \hline
        \textbf{AOI} & \textbf{Phrase} & \textbf{Effect Term} & \textbf{Effect} & \textbf{Effect Meaning} \\
        \hline
        Target & 1 & Duration d' $\times$ Duration & Negative & Listeners with lower duration sensitivity fixated less on short-duration focus-marked words. \\
        Target & 1 & Formants d' $\times$ Word Stress & Positive & Higher formant sensitivity led to increased fixations on stressed words. \\
        Target & 1 & Pitch d' $\times$ Pitch Range & Positive & Listeners with greater pitch sensitivity fixated more when pitch range was wider. \\
        \hline
        Target & 2 & Stress & Negative & Listeners fixated less on targets in verb-focused sentences. \\
        Target & 2 & Duration & Negative & Shorter words resulted in fewer fixations on the target. \\
        Target & 2 & Duration d' $\times$ Duration & Negative & Fewer fixation on short-duration words for listeners with lower duration sensitivity. \\
        Target & 2 & Pitch Range $\times$ Melody & Negative & Listeners with weaker melodic reproduction ability fixated less when pitch range was smaller. \\
        \hline
        Competitor & 1 & Duration & Negative & Fewer competitor fixations for words with shorter duration. \\
        Competitor & 1 & L1 & Positive & L1 Dutch-L2 English speakers showed greater fixations. \\
        Competitor & 1 & Word Stress & Positive & Words with higher stress prominence attracted more competitor fixations. \\
        Competitor & 1 & L1 $\times$ Stress & Negative & Fewer fixations by L1 Dutch-L2 English speakers for stressed words. \\
        Competitor & 1 & Duration d' $\times$ Duration & Positive & More competitor fixations when words had longer duration. \\
        Competitor & 1 & Formants d' $\times$ Word Stress & Positive & Higher formant sensitivity led to increased competitor fixations for stressed words. \\
        Competitor & 1 & Melody $\times$ Pitch Range & Positive & Higher melodic ability led to increased fixations on competitors when pitch range was wider. \\
        \hline
        Competitor & 2 & Stress & Positive & Focus-marking increased competitor fixations. \\
        Competitor & 2 & Duration & Positive & More competitor fixations for words with shorter duration. \\
        Competitor & 2 & Pitch d' & Positive & Higher pitch sensitivity led to increased competitor fixations. \\
        Competitor & 2 & Pitch d' $\times$ Pitch Range & Positive & Competitor fixations increased when pitch range was wider for listeners with higher pitch sensitivity. \\
        Competitor & 2 & L1 $\times$ Stress & Negative & Fewer fixations for L1 English speakers during verb focus. \\
        \hline
    \end{tabular}
    }
    \caption{Summary of fixation patterns across phrase models, organized by AOI, phrase, effect term, direction, and meaning.}
    \label{tab:fixation_patterns}
\end{table}

