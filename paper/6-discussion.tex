This paper is an attempt at mitigating the risks involved in any potential replication while also ensuring that the replication results, no matter what the outcome, advance our knowledge, however small. 

\subsection{Web-based replication of \cite{Ge2021}}

\subsubsection{Replication vs. Fidelity: a faithful reproduction with minimal changes}

\cite{Ge2021} investigated how native (L1 English speakers) and non-native (L1 Dutch) speakers process focus in English sentences using the visual world paradigm. Their study reported three central findings. 1) Focus processing advantage for L1 English speakers. That is, L1 English speakers exhibit significantly earlier fixations to focus-alternative referents compared to Dutch speakers. This finding was taken to mean that native speakers integrate prosodic cues more efficiently in real-time sentence processing. A second finding of \cite{Ge2021} was that L2 learners show a delay in focus driven eye-fixations. This delay was attribute to L2 speakers' difficulties in mapping prosodic cues to meaning. Meaning that L2 learners tend to focus on segmental information more than prosodic information especially if their L1 does not use focus (e.g., Dutch). 

Our fidelity based replication here did not find evidence  of either of these key findings. For the first, we simply did not find that native speakers use prosodic information earlier or more efficiently. In this way, the effect was not replicated. One could argue that we did find partial evidence that L2 learners of English have delayed focus processing for English focus. However, this result is tenuous. While both our replication and \cite{Ge2021} showed evidence of differences in eye-movement patterns during focus processing, our results deviated in almost every possible way. In some ways, we found the opposite of the original study. That is, Dutch speakers have more competitor focus fixation than English speakers earlier on. While our results revealed some statistically significant differences in eye fixation patterns across object focused and verb focused sentences, the nature of these results did not align with \cite{Ge2021}. These discrepancies raise important questions about the role of methodological differences, population-level effects, and statistical interpretation in replication research. 

There are at least three possible reasons for these discrepancies: differences between data collection methods, false positives-negatives, and group level differences. As far as difference in data collection methods. \cite{Ge2021} conducted their study in a controlled lab setting, using high-precision eye-trackers with different frame rates depending on the participant group. L1 Dutch speakers were tested in the Netherlands using an eye-tracker with a higher sampling rate (500 Hz), whereas L1 English speakers were tested in Hong Kong with a lower sampling rate (300 Hz). This difference in eye-tracking setups means that the timing resolution of fixations was not uniform across groups, potentially influencing the reported processing differences. In contrast, our study utilized web-based eye-tracking, which introduces additional variability due to differences in screen resolution, calibration accuracy, and participant engagement. But even with these differences our participants all used webcams and any variability in fixation capturing was random across groups (low and high frame rates in both groups). The choice to move to web-based eye-tracking here then can be seen as a 双刃剑 (double edged sword). Yes, web-based eye-tracking hass lower frame rate and more variable. But the variability is consitent across randomly sampled populations (population is discusssed below). As a result, it is possible that our inability to replicate their findings is at least partially due to the differences in data collection methods rather than differences in cognitive processing.

Secondly, the presence of false positives and false negatives in either our study or \citep{Ge2021} could contribute to discrepancies between our results and theirs. While our sample sizes (Dutch = 31, English = 61) are comparable to those of \citep{Ge2021} (Dutch = 35, English = 40), the number of statistical tests conducted per language (9) increases the likelihood of Type I errors. Given this, there is a 59.34\% probability of obtaining at least one false positive in both studies, making it crucial to interpret significant findings with caution. The risk of false negatives is more complex to quantify. If prosodic effects exist but are small, our sample sizes may be underpowered to detect them, leading to Type II errors. However, estimating this risk is particularly challenging due to the lack of established effect sizes for prosody in focus processing. Since this field is still emerging, future research should aim to establish reliable effect size estimates to improve statistical power calculations and minimize both false positives and false negatives.

Third, although we followed a similar recruitment strategy to \citep{Ge2021}, differences in L2 proficiency, English exposure, or individual cognitive-perceptual abilities between participant samples may have contributed to the discrepancies in findings. Additionally, while \citep{Ge2021} labeled their L1 English group as a control, this group was recruited in Hong Kong, meaning they may not be directly comparable to other L1 English-speaking populations. Furthermore, exposure to or experience with tone languages like Cantonese and Mandarin may have influenced English focus processing patterns. The greater reliance on lexical pitch in these languages could have heightened participants’ attentional allocation to prosodic cues in English, potentially enhancing sensitivity to focus marking. However this is pure speculation as linguistic experience of participants in the \citep{Ge2021} was limited (i.e., exchange students with limited or no proficiency). However, if such differences in prosodic attention played a role in the original findings, they may not generalize across populations with different linguistic backgrounds, helping to explain why our study did not replicate the same effects. This suggests that focus processing may be more variable than previously assumed, and that group-level trends in prosodic cue integration may not be as stable across studies. Further disuccusion of individual difference can be found in our extension.

In sum, while we found statistically significant effects in our data, it is important to emphasize that our fidelity based replication did not test for direct interactions between language groups. This choice was in the name of fidelity due to the fact that Ge et al. (2021) conducted separate statistical analyses for each group, meaning that group-level differences were inferred rather than explicitly tested via interaction effects. The fact that our study did not replicate these key findings from \citep{Ge2021} highlights the complexities of replication in psycholinguistics and second language research. This methodological limitation raises concerns about the robustness of previously reported L1-L2 differences, particularly given the potential for false positives when conducting multiple separate statistical tests. If group differences were found in one study but not another, this does not necessarily indicate a true underlying cognitive difference; rather, it could be the result of analytical choices or sample variability. This is particularly important when working with small effect sizes, where statistical significance may not always equate to meaningful or replicable findings. Given these considerations, our fidelity replication highlights the importance of directly testing interactions which we in term do in the next section.


\subsection{Replication vs. refinement: Balancing rigor and practicality}

A key limitation of both our fidelity replication and the original study by \citep{Ge2021} is the separation of L1 English and L1 Dutch participants into independent statistical models. While this approach may have been necessary in \citep{Ge2021} due to differences in eye-tracking hardware across participant groups, it prevented a direct statistical test of whether group differences were robust or whether they emerged as an artifact of separate analyses. To address this, we refined our analysis by adopting a time-sensitive model that accounts for gradual fixation changes rather than treating time as discrete bins. Using Generalized Additive Models (GAMs), we assessed whether prosodic effects emerged immediately or developed over time. This refined approach allowed us to test whether L1 speakers truly showed an early fixation advantage and whether L2 speakers were uniformly delayed in processing prosodic focus cues.

The refined analysis produced three key findings that deepen our understanding of prosodic cue integration. First, our results challenge the assumption that L1 English speakers use prosodic focus cues immediately. In \citep{Ge2021}, English speakers were reported to fixate on focus-marked referents earlier than Dutch speakers, suggesting that native speakers efficiently integrate prosodic cues from the outset. However, our target model and competitor model for phrase one did not find a main effect of L1 early in processing. Instead, fixation patterns gradually increased over time, as shown by a main effect of time with object focus leading to greater fixations. This means that fixations to the target did not shift immediately following prosodic cues, but rather developed dynamically, suggesting that previous studies may have overestimated the immediacy of prosodic integration.

Second, rather than revealing a uniform L2 delay, our results show that Dutch speakers were not consistently slower than English speakers in processing prosodic focus cues. In fact, the interaction between stress and L1 may indicate this difference is negative for both target and competitors models indicating less english looks overall rather than an L1 advantage. This contradicts the expectation that Dutch participants should show a general delay in focus-driven fixations. However, a three-way interaction between time, L1, and stress revealed that L1 English speakers gradually developed a fixation preference for the target in verb-stressed sentences. This suggests that while Dutch speakers did not initially lag behind, L1 English speakers showed a more stable late-stage fixation pattern over the first phrase. In other words, the native speaker advantage may be one of long-term consistency rather than early efficiency.

Third, our refined approach revealed that group-level differences were previously overstated due to statistical modeling choices. In \citep{Ge2021}, L1 and L2 differences were inferred from separate statistical models rather than directly tested as an interaction. Our unified model revealed a negative interaction between L1 and stress in the first phrase competitor model, indicating that English speakers showed fewer fixations to the competitor during verb-stressed sentences. At the same time, a negative interaction between time and stress showed that fixations to the competitor decreased more sharply for English speakers as time progressed. These findings indicate that differences between L1 and L2 participants were not static processing deficits but instead emerged gradually through dynamic changes in fixation patterns over time.

These results have key implications for how we interpret prosodic focus effects in L1 vs. L2 processing. If focus-driven fixations are not immediate, then L2 delays may not reflect an inherent cognitive limitation but rather a difference in cue weighting over time. Similarly, if L1 advantages emerge later in processing rather than earlier, then the assumption that native speakers automatically integrate prosodic cues faster may need to be reconsidered. Finally, the fact that L1-L2 differences were highly sensitive to analytical approach underscores the need for caution when making strong claims about group differences based on discrete time bins.

Unlike our fidelity analysis, our refined analysis does not overturn the key findings of \citep{Ge2021}, but it suggests that their interpretation may have been overly rigid. Rather than a clear and immediate L1 advantage, our results indicate that prosodic cue integration is dynamic, shifting over time rather than appearing as a static group-level effect. By adopting an approach that accounts for these time-dependent changes, we provide a more nuanced understanding of how both native and non-native speakers process prosodic focus cues.

\subsection{Exploratory Extension}

The replication and refinement analyses established that prosodic focus effects were not robust across all conditions, raising questions about the factors that shape fixation patterns in real-time speech processing. While methodological fidelity and statistical rigor allowed us to test for consistency with \citep{Ge2021}, this approach did not account for the inherent variability in acoustic properties of speech and individual differences in perception. The exploratory extension presented here aims to address these gaps by examining how fine-grained acoustic variation and individual perceptual abilities influence prosodic focus processing. By moving beyond group-level comparisons, we assess whether listeners’ auditory sensitivity, musical abilities, and cognitive resources play a role in determining how focus-marked speech is processed.

Acoustics and Acoustic Sensitivity

The acoustic properties of the stimuli played a significant role in shaping fixation patterns, revealing that prosodic focus effects were not uniform across speech tokens. Rather than reflecting a simple effect of focus marking, word duration, pitch range, and stress prominence interacted with individual acoustic sensitivity, determining how listeners allocated attention to different referents.

In the first phrase target model, a negative effect of duration indicated that shorter words resulted in fewer fixations to the target. This suggests that listeners were more likely to use prosodic cues to direct their attention when focus-marked words were acoustically prominent, particularly when their duration was lengthened. Similarly, in the first phrase competitor model, a positive interaction between duration d’ (individual sensitivity to duration contrasts) and word duration indicated that listeners with higher duration sensitivity were more likely to fixate on competitors when words had longer durations. This suggests that perception of prosodic prominence is contingent on both the acoustic properties of speech and the listener’s ability to track duration-based prominence cues.

Pitch variation also influenced fixation behavior. In the second phrase competitor model, a positive interaction between pitch d’ and pitch range  indicated that listeners with greater pitch sensitivity were more likely to fixate on competitors when words had a larger pitch range. This suggests that prosodic focus effects were enhanced for listeners who could perceive fine-grained pitch variations, leading them to shift fixations toward competitors when pitch was exaggerated.Together, these findings highlight that prosodic focus processing is not solely driven by categorical stress marking; but instead, the acoustic properties of speech interact with individual auditory sensitivity to shape fixation patterns dynamically. See summary table \ref{tab:fixation_patterns}

\begin{table}[h]
    \centering
    \renewcommand{\arraystretch}{1.3}
    \resizebox{\textwidth}{!}{
    \begin{tabular}{|l|c|l|c|l|}
        \hline
        \textbf{AOI} & \textbf{Phrase} & \textbf{Effect Term} & \textbf{Effect} & \textbf{Effect Meaning} \\
        \hline
        Target & 1 & Duration d' $\times$ Duration & Negative & Listeners with lower duration sensitivity fixated less on short-duration focus-marked words. \\
        Target & 1 & Formants d' $\times$ Word Stress & Positive & Higher formant sensitivity led to increased fixations on stressed words. \\
        Target & 1 & Pitch d' $\times$ Pitch Range & Positive & Listeners with greater pitch sensitivity fixated more when pitch range was wider. \\
        \hline
        Target & 2 & Stress & Negative & Listeners fixated less on targets in verb-focused sentences. \\
        Target & 2 & Duration & Negative & Shorter words resulted in fewer fixations on the target. \\
        Target & 2 & Duration d' $\times$ Duration & Negative & Less fixation on short-duration words for listeners with lower duration sensitivity. \\
        Target & 2 & Pitch Range $\times$ Melody & Negative & Listeners with weaker melodic ability fixated less when pitch range was compressed. \\
        \hline
        Competitor & 1 & Duration & negative & less competitor fixations for words with shorter duration \\
        Competitor & 1 & L1 & positive & Dutch speakers more fixations. \\
        Competitor & 1 & Word Stress & Positive & Words with higher stress prominence attracted competitor fixations. \\
        Competitor & 1 & L1 $\times$ Stress & Negative & Interaction between L1 and stress influenced fixations. \\
        Competitor & 1 & Duration d' $\times$ Duration & Positive & More competitor fixations when words had longer duration. \\
        Competitor & 1 & Formants d' $\times$ Word Stress & Positive & Higher formant sensitivity led to increased competitor fixations for stressed words. \\
        Competitor & 1 & Melody $\times$ Pitch Range & Positive & Higher melodic ability led to increased fixations on competitors when pitch range was wider. \\
        \hline
        Competitor & 2 & Stress & Positive & Focus-marking increased competitor fixations. \\
        Competitor & 2 & Duration & Positive & More competitor fixations for words with shorter duration. \\
        Competitor & 2 & Pitch d' & Positive & Higher pitch sensitivity led to increased competitor fixations. \\
        Competitor & 2 & Pitch d' $\times$ Pitch Range & Positive & Competitor fixations increased when pitch range was wider for listeners with higher pitch sensitivity. \\
        Competitor & 2 & L1 $\times$ Stress & Negative & Less fixations for English speakers during verb focus. \\
        \hline
    \end{tabular}
    }
    \caption{Summary of fixation patterns across phrase models, organized by AOI, phrase, effect term, direction, and meaning.}
    \label{tab:fixation_patterns}
\end{table}

Acoustics and Rhythm/Melody Abilities

Beyond basic acoustic sensitivity, listeners’ rhythm and melody abilities further influenced how they processed prosodic focus. These higher-level perceptual skills—often linked to musical training—moderated the extent to which fixations aligned with prosodic prominence in the speech signal especially at the later parts of the sentence (second phrases).

In the first phrase target model, a negative interaction between pitch range and melodic ability revealed that listeners with lower melodic abilities exhibited fewer fixations to the target when pitch range was compressed. This suggests that prosodic focus effects were stronger for listeners who could track pitch variation with greater precision—when the pitch range was reduced, those with weaker melodic skills failed to fixate as reliably on focus-marked words. A similar trend was observed in the first phrase competitor model, where a positive interaction between melody and pitch range indicated that listeners with greater melodic ability fixated more on competitors when pitch range was exaggerated. This suggests that musically skilled listeners were more sensitive to prosodic variation, shifting their attention in response to subtle acoustic differences.

These findings emphasize that prosodic focus effects are not purely linguistic but also linked to auditory experience and perceptual skill. Rather than treating prosody as a strictly phonological phenomenon, these results align with research showing that musical abilities contribute to real-time speech processing, particularly in languages where prosody conveys contrastive meaning.

Cognitive Factors: Working Memory, Cognitive Control, and LexTALE

Despite their hypothesized role in prosodic processing, cognitive factors such as working memory, cognitive control, and lexical knowledge (LexTALE) did not significantly predict fixation behavior. Unlike acoustic perception and musical abilities, which directly shaped fixation patterns, higher-order cognitive measures were not reliable predictors of how listeners processed prosodic focus cues.

No significant effects were found for working memory capacity, cognitive control, or LexTALE scores, suggesting that prosodic focus integration relies more on perceptual mechanisms than on domain-general cognitive abilities. This challenges assumptions that higher cognitive control or stronger lexical representations necessarily enhance prosodic processing. The LexTALE results must be interpretted with caution, however. The Dutch group scored very high on the lexTALE task, which may indicate a ceiling effect. It is not clear if a wider range of English proficiency may leader to more varying results. The evidence of our extend exploration supports a model in which real-time fixation shifts are shaped more by perceptual abilities than by cognitive resources.


\subsection{Moving Beyond L1-L2 Differences}

In sum, our exploration findings extend our replication and refinement by demonstrating that prosodic focus processing is not solely driven by language background (L1 vs. L2) but is shaped by a combination of acoustic properties of stimuli and listener-specific traits. While \citep{Ge2021} attributed differences in focus processing to L1 effects, our results suggest that listener-specific auditory and cognitive abilities play a more central role than previously assumed.

First, fixations to focus-marked referents were systematically modulated by acoustic properties such as pitch range and word duration, rather than being an invariant effect of prosodic marking. This suggests that prosodic processing is highly stimulus-dependent, meaning that prosodic focus effects are not universal but contingent on specific acoustic realizations of stress and prominence.

Second, L1-L2 differences in prosodic processing were less stable than assumed. While previous studies suggested that non-native listeners exhibit a uniform delay, our results show that fixation patterns are better predicted by individual differences in auditory sensitivity tied to specific acoustic propoerties of the sitmuli \cite{xie2023adaptive}. In other words, prosodic processing is highly individualized, and L2 delays may reflect perceptual and cognitive variability rather than a fixed group-level effect.

Finally, our results highlight the importance of including individual differences in speech perception research L1 and L2 research. By integrating auditory perception, cognitive control, and lexical proficiency, we provide a more nuanced account of prosodic focus processing, moving beyond broad group-level contrasts to uncover how individual listeners dynamically adapt to prosodic cues in real time. Traditional L1/L2 comparisons often treat native language effects as a black box, attributing differences in speech processing to categorical group distinctions without specifying the underlying mechanisms. However, this approach lacks parsimony, as it assumes that L1 itself is the explanatory factor rather than identifying the perceptual and cognitive mechanisms that drive these differences. Our findings suggest that focus processing is better explained through mechanistic factors such as individual variation in acoustic sensitivity (e.g., pitch d$’$, duration d$’$) and cognitive control, rather than broad L1 effects. This more parsimonious framework accounts for why some L2 speakers approach native-like processing while some L1 speakers do not consistently exhibit the expected pattern. Rather than treating L1 effects as static, we show that they emerge from individual differences in sensitivity to speech cues, aligning with adaptive models of speech perception (Xie et al., 2023). Future work should move beyond categorical L1/L2 distinctions and instead focus on how variation in perceptual and cognitive traits modulates speech processing across speakers, providing a more mechanistic and theoretically grounded explanation of prosodic processing.


\subsubsection{Limitations}

While this study provides valuable insights into the replication of prosodic focus processing, several limitations should be acknowledged. First, the use of web-based eye-tracking introduces variability in gaze data due to differences in participant screen sizes, webcam qualities, and environmental conditions. Although we implemented stringent calibration procedures and data filtering, the lower sampling rate compared to lab-based studies may have affected the temporal precision of fixation patterns. Future research should further validate web-based methods against high-resolution lab-based eye-tracking.

Second, our replication was limited by the absence of L1 Cantonese participants due to geopolitical constraints. Since the original study found differences between L1 Dutch and L1 Cantonese speakers, we were unable to assess whether web-based methods reproduce these cross-linguistic effects. Additionally, given that Dutch participants were recruited via Prolific, individual differences in language exposure and proficiency may have contributed to variation in results. Future studies should explore whether participant recruitment methods influence replication outcomes.

Third, while our statistical approach using Generalized Additive Models (GAMs) and LASSO-based feature selection provided a more nuanced analysis of fixation dynamics, it also introduced complexity in model interpretation. The inclusion of multiple individual difference measures allowed for a richer understanding of variability in focus processing, but further replications with larger and more diverse samples are needed to determine the generalizability of these findings.

Finally, our study highlights the importance of incorporating acoustic properties into analyses of focus processing. Our stimuli were constructed sentences rather than fully natural speech, which may have limited the variability in acoustic cues available to listeners. A more comprehensive approach could involve testing a broader, naturally varying acoustic space to assess whether prosodic focus processing is influenced by fine-grained phonetic variation. Future work should investigate how individual differences interact with prosodic variation across a more ecologically valid range of speech input.

By addressing these limitations, future research can refine our understanding of how prosodic focus is processed across different speaker populations and methodological contexts.



