This paper is an attempt at mitigating the risks involved in any potential replication while also ensuring that the replication results, no matter what the outcome, advance our knowledge, however small. 

\subsection{Web-based replication of Ge et al. (2016)}

\subsubsection{Replication vs. refinement: Balancing rigor and practicality}


The seperation of language between models keeps a full comparison of behavior impossible. It is essential to note, however, that this full comparison may not have been possible in \cite{Ge2021}, beceause of differences in eye-trackers across populations. 

Studies chosen for replication often fail to clearly delineate between confirmatory analyses that test an explicit hypothesis and exploratory analyses that are post hoc in nature (see \cite{Yanai2020} for discussion of daytime- and nighttime-science). This lack of transparency complicates replication efforts, particularly when determining whether replications should adhere strictly to the original analytical approach, even if it was statistically less rigorous, or refine the methodology in alignment with current best practices, which in turn introduces unintended differences that can affect the comparability between the replication and the original study \cite{mcmanus2022replication}.


\subsection{Replication vs. refinement: Balancing rigor and practicality}


\subsection{Exploratory Extension}

\subsection{Contributions to L2 theory}

\subsubsection{Limitations}



