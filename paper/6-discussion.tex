%This paper is an attempt at mitigating the risks involved in any potential replication while also ensuring that the replication results, no matter what the outcome, advance our knowledge, however small. 

\section{Discussion}
\subsection{Replication vs. Fidelity: a faithful reproduction with minimal changes}

\cite{ge2021a} investigated how L1 English speakers and L1 Dutch-L2 English speakers process focus in English \textit{only}-sentences using the "look-and-listen" visual world paradigm. Their study reported two key findings that we set out to replicate: 1) L1 English speakers exhibit significantly earlier fixations to focus-alternative referents compared to L1 Dutch-L2 English speakers because L2 learners struggle to integrate multiple interfaces in real time. 2) L2 learners struggle to acquire focus prosody in the target language in accordance with the Prosodic-Learning Interference Hypothesis \parencite{tremblay2016effects, tremblay2021re}, which states that L2 learning of prosodic cues is more difficult when the L1 and L2 use similar prosodic cues as in Dutch-English.

Our fidelity-based replication found some limited evidence for these findings. Although we did not find the same effects in the same time bins that \cite{ge2021a} reported, we did find that L1 speakers used prosodic information relatively more efficiently than L2 speakers. That is, we observed looks to the focus competitor during ``not" for L1 speakers but not for L2 speakers. We also observed for the L2 group relatively early competitor effects (before ``not" or other prosodic information was available) and relatively late target looks. Thus, L1 Dutch-L2 English speakers seemed to struggle to integrate multiple interfaces in real time. One interpretation of our results--that align with \cite{ge2021a}'s--is that the timing of L1 and L2 looks does, in fact, differ. 

Our fidelity-based replication did not test for direct interactions between language groups. This was because \cite{ge2021a} conducted separate statistical analyzes for each group, which means that group-level differences were inferred rather than explicitly tested via interaction effects. The fact that our study did not replicate these key findings from \cite{ge2021a} highlights the complexities of replication in psycholinguistics. This methodological limitation raises concerns about the robustness of previously reported L1-L2 differences, particularly given the potential for false positives when conducting multiple separate statistical tests. If group differences were found in one study but not another, this does not necessarily indicate a true underlying cognitive difference; rather, it could be the result of analytical choices or sample variability. This is particularly important when working with small effect sizes, where statistical significance may not always equate to meaningful or replicable findings (see Limitations below). Given these considerations, our fidelity replication highlights the importance of directly testing interactions, which we look at next in our refinement.


\subsection{Replication vs. refinement: Balancing rigor and practicality}

A key limitation of both our fidelity replication and the original study by \cite{ge2021a} is the separation of L1 English and L1 Dutch participants into independent statistical models. Whereas this approach may have been necessary in \cite{ge2021a} due to differences in eye-tracking sampling rates across participant groups, it prevented a direct statistical test of whether group differences were robust or whether they emerged as an artifact of separate analyses. To address this, we refined our analysis by adopting a time-sensitive model that accounts for gradual fixation changes across each phrase rather than treating time as discrete bins. Using Generalized Additive Models (GAMs), we assessed whether prosodic effects emerged immediately or developed over time. This refined approach allowed us to test whether L1 speakers truly showed an early fixation advantage and whether L2 speakers were uniformly delayed in processing prosodic focus cues.

The refined analysis produced three key findings that deepen our understanding of prosodic cue integration. First, our target model and competitor model for phrase one indicated that fixation patterns gradually increased over time, as shown by an effect of time with object focus leading to greater fixations. This means that fixations to the target did not shift immediately following prosodic cues, but rather developed dynamically over time as multiple levels of linguistic information was integrated \parencite[see][]{Lambrecht1994}.

Second, rather than revealing a uniform L2 delay, our results show that L1 Dutch-L2 English speakers were not consistently slower than L1 English speakers in processing prosodic focus cues. In fact, the interaction between stress and L1 indicates that this difference is negative for both target and competitors models indicating fewer looks overall rather than an L1 advantage. A three-way interaction between time, L1, and stress further revealed that L1 English speakers gradually developed a fixation preference for the target in verb-stressed sentences whereas L2 speakers did not. This suggests that during the first phrase, L1 English speakers showed a more stable late-stage fixation pattern. In other words, the L1 speaker advantage may be one of long-term consistency rather than early efficiency.

Third, our refined approach revealed that group-level differences may have been previously overstated due to statistical modeling choices. In \cite{ge2021a}, L1 and L2 differences were inferred from separate statistical models rather than directly tested as an interaction. Our unified model revealed a negative interaction between L1 and stress in the first phrase competitor model, indicating that L1 English speakers showed fewer fixations to the competitor during verb-stressed sentences. At the same time, a negative interaction between time and stress showed that fixations to the competitor decreased more sharply for L1 English speakers as time progressed. These findings indicate that differences between L1 and L2 participants were not static processing deficits but instead emerged gradually through dynamic changes in fixation patterns over time.

Importantly, our refined analysis does not overturn the key findings of \cite{ge2021a}, rather it suggests that their interpretation may have been overly rigid. As opposed to a clear and immediate L1 advantage, our results indicate that prosodic cue integration is dynamic, shifting over time rather than appearing as a static group-level effect. By adopting an approach that accounts for these time-dependent changes, we provide a more nuanced understanding of how both L1 and L2 speakers process prosodic focus cues.

\subsection{Exploratory extension: Where we go from here}

 Whereas methodological fidelity and statistical refinement allowed us to test for consistency with \cite{ge2021a}, we set out to explore how fine-grained acoustic-phonetic properties of the stimuli affect participants' eye movements (EE1), and whether working memory (EE2), cognitive control (EE3), English lexical proficiency (EE4), perceptual auditory sensitivity (EE5), and auditory motor reproduction (EE6) affect participants' eye movements.
 
\subsubsection{Acoustics and acoustic sensitivity}

As expected, the acoustic properties of the stimuli played a significant role in shaping fixation patterns \parencite{magnuson2019fixations}. Across target and competitor analyses and during both the first and second phrase, we repeatedly found evidence for participants using duration, stress (spectral tilt), and pitch range information in real-time in line with previous acoustic findings \cite{Breen2010,baumann2018makes}. In addition, we found evidence that individual differences in acoustic sensitivity interacted with the acoustics of the stimuli to further influence eye movements (see Table \ref{tab:fixation_patterns}). For example, in the first phrase competitor model, a positive interaction between duration d$’$ (individual sensitivity to duration contrasts) and word duration indicated that listeners with higher duration sensitivity were more likely to fixate on competitors when words had longer durations. This suggests that perception of prosodic prominence is contingent on both the acoustic properties of speech and the listener’s ability to track duration-based prominence cues. Similarly, pitch variation also influenced fixation behavior. In the second phrase competitor model, a positive interaction between pitch d’ and pitch range  indicated that listeners with greater pitch sensitivity were more likely to fixate on competitors when words had a larger pitch range. This suggests that prosodic focus effects were enhanced for listeners who could perceive fine-grained pitch variations, leading them to shift fixations toward competitors when pitch was exaggerated. These results are in line with a growing body of research that shows that prosodic processing is not driven solely by categorical acoustic marking; rather, the acoustic properties of speech interact with individual auditory sensitivity to shape fixation patterns dynamically \parencite{roy2017individual,ppcc,bramlett_wiener_24_speechprosody,jansen2023influence}.

\subsubsection{Acoustics and auditory motor reproduction abilities}

Beyond basic acoustic sensitivity, listeners’ melody reproduction abilities further influenced how they processed prosodic focus. These higher-level perceptual skills moderated the extent to which fixations aligned with prosodic prominence in the speech signal especially at the later parts of the sentence (second phrases). In the first phrase target model, a negative interaction between pitch range and melodic reproduction ability revealed that listeners with lower melodic reproduction abilities exhibited fewer fixations to the target when pitch range was compressed. This suggests that prosodic focus effects were stronger for listeners who could reproduce pitch variation with greater precision—when the pitch range was reduced, those with weaker melodic skills failed to fixate as reliably on focus-marked words. A similar trend was observed in the first phrase competitor model, where a positive interaction between melody reproduction and pitch range indicated that listeners with greater melodic ability fixated more on competitors when pitch range was exaggerated. This suggests that musically skilled listeners were more sensitive to prosodic variation, shifting their attention in response to subtle acoustic differences. This strongly supports \cite{jansen2023influence}, which found positive effects of music on perception of L2 focus prosody and, more generally, supports claims for domain-general auditory processing \parencite{saito2022does, bramlett_wiener_24_speechprosody, bakkouche2025effects, Kachlicka_Saito_Tierney_2019}.


\subsubsection{Cognitive factors: Working wemory, cognitive control, and LexTALE}

Despite their hypothesized role in prosodic processing, working memory, cognitive control, and lexical proficiency (LexTALE) did not significantly predict fixation behavior in our study. This does not necessarily mean that these predictors do not contribute to focus processing. There are at least three explanations for our null results. First, we did not have a large enough sample. This is probably true for our L1 Dutch-L2 English participants (N = 27). Second, our specific population was at or near ceiling in many tasks, such as LexTALE. The L1 Dutch-L2 English group scored very high on the LexTALE task, with many participants actually outperforming L1 English participants (though it is interesting to note despite this high lexical proficiency, the L2 participants' eye movements still indicate less efficient real-time processing). It is not clear if a wider range of English proficiency may lead to more varying results. Third, the look-and-listen task may require fewer cognitive resources than a word recognition task that involves multiple choices and greater attention. We tentatively conclude that our look-and-listen results were shaped more by perceptual and motor reproduction abilities than by cognitive resources.


\subsection{Moving beyond L1-L2 differences}

In sum, our exploration findings extend our replication and refinement by demonstrating that prosodic focus processing is not solely driven by language background (L1 vs. L2) but is shaped by a combination of acoustic properties of stimuli and listener-specific traits. While \cite{ge2021a} attributed differences in focus processing to L1 effects, our results suggest that domain-general, listener-specific auditory and motor abilities play a more central role than previously assumed \parencite{saito2022does, bramlett_wiener_24_speechprosody, bakkouche2025effects, Kachlicka_Saito_Tierney_2019}. While previous studies suggested that L2 listeners exhibit a uniform delay, our results show that fixation patterns are better predicted by individual differences in auditory sensitivity tied to specific acoustic properties of the stimuli \parencite{xie2023adaptive}. In other words, prosodic processing is highly individualized, and L2 delays may reflect perceptual and cognitive variability rather than a fixed group-level effect. 

Finally, our results highlight the importance of including individual differences in L1 and L2 speech perception research. Traditional L1/L2 comparisons often treat L1 monolinguals as a control \parencite{rothman2023monolingual}, attributing differences in speech processing to categorical group distinctions without specifying the underlying mechanisms. However, this approach lacks parsimony, as it assumes that the L1 itself is the explanatory factor rather than identifying the perceptual and cognitive mechanisms that drive these differences. Our findings suggest that focus processing is better explained through mechanistic factors such as individual variation in acoustic sensitivity (e.g., pitch d$’$, duration d$’$) rather than broad L1 effects. This more parsimonious framework accounts for why some L2 speakers approach L1-like processing while some L1 speakers do not consistently exhibit the expected pattern. Rather than treating L1 effects as static, we show that they emerge from individual differences in sensitivity to speech cues, aligning with adaptive models of speech perception \parencite{xie2023adaptive}. 


\subsection{Limitations}

There are at least three notable deviations from our study and \cite{ge2021a}: differences between data collection methods, false positives-negatives, and group-level differences. 

Regarding the difference in data collection methods, \cite{ge2021a} conducted their study in a controlled lab setting, using high-precision eye-trackers with different frame rates depending on the participant group. L1 Dutch speakers were tested in the Netherlands using an eye-tracker with a 500 Hz sampling rate, whereas L1 English speakers were tested in Hong Kong with a 300 Hz sampling rate. The use of web-based eye-tracking introduces variability in gaze data due to differences in participant screen sizes, webcam qualities, and environmental conditions. Although we implemented stringent calibration procedures and data filtering, our web-based sample had variable frame rates ranging from 5 Hz to 60 Hz, as is common in web-based eye-tracking \parencite{Vos_2022,AOW}. The lower sampling rate compared to lab-based studies may have affected the temporal precision of fixation patterns.

Secondly, the presence of false positives and false negatives in either our study or \cite{ge2021a} could contribute to discrepancies between our results and theirs. While our sample sizes (Dutch = 27, English = 61) are comparable to those of \cite{ge2021a} (Dutch = 35, English = 40), the number of statistical tests conducted per language (9) increases the likelihood of Type I errors. Given this, there is a 59.34\% probability of obtaining at least one false positive in both studies, making it crucial to interpret significant findings with caution. The risk of false negatives is more complex to quantify. If prosodic effects exist but are small, our sample sizes may be underpowered to detect them, leading to Type II errors. However, estimating this risk is particularly challenging due to the lack of established effect sizes for prosody in focus processing. Since this field is still emerging, future research should aim to establish reliable effect size estimates to improve statistical power calculations and minimize both false positives and false negatives. To clarify, we are not claiming that all effects across the two studies are merely false positives but rather that being able to separate the false positives from real effects is not feasible at this stage. Similarly, while our statistical approach using Generalized Additive Models and LASSO-based feature selection provided a more nuanced analysis of fixation dynamics, it also introduced complexity in model interpretation. The inclusion of multiple individual difference measures allowed for a richer understanding of variability in focus processing, but further replications with larger and more diverse samples are needed to determine the generalizability of these findings.

Third, although we followed a recruitment strategy similar to \cite{ge2021a}, differences in L2 proficiency, English exposure, or individual cognitive-perceptual abilities between participant samples may have contributed to the discrepancies in findings. It should be noted that \cite{ge2021a} recruited their L1 English speakers in Hong Kong, which means that they may not be directly comparable to other L1 English speaking populations. Exposure to or experience with tone languages like Cantonese and Mandarin may have influenced English focus processing patterns. The greater reliance on lexical pitch in these languages could have heightened participants’ attentional allocation to prosodic cues in English, potentially enhancing sensitivity to focus marking. Indeed, there is evidence that L1 tonal experience improves L2 English stress perception to behavior better than that of L1 listeners \parencite{choi2019better, choi2021cantonese}. By contrast, our L1 participants were recruited online through prolific rather than through university cohorts. Such differences broaden the demographic profile but also 
introduce more heterogeneity. \hl{This is common in replication studies and should be considered alongside modality and analytical factors when interpreting variation between the current study and Ge et al. (2021).}


