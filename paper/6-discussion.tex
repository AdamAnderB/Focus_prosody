This paper is an attempt at mitigating the risks involved in any potential replication while also ensuring that the replication results, no matter what the outcome, advance our knowledge, however small. 

\subsection{Web-based replication of Ge et al. (2016)}

\subsubsection{Replication vs. Fidelity: a faithful reproduction with minimal changes}

\cite{Ge2021} investigated how native (L1 English speakers) and non-native (L1 Dutch) speakers process focus in English sentences using the visual world paradigm. Their study reported three central findings. 1) Focus processing advantage for L1 English speakers. That is, L1 English speakers exhibit significantly earlier fixations to focus-alternative referents compared to Dutch speakers. This finding was taken to mean that native speakers integrate prosodic cues more efficiently in real-time sentence processing. A second finding of \citep{Ge2021} was that L2 learners show a delay in focus driven eye-fixations. This delay was attricubte to L2 speakers difficulites in mapping prosodic cues to meaning. Meaning that L2 learners tend to focus on segmental information more than prosdic information especially if their L1 does not use focus (e.g. Dutch). 

Our fidelity based replication here did not find evidence  of either of these key findings. For the first, we simply did not find that native speakeers use prosdic information earlier or more efficiently. In this way the effect was not replicated. One could argue that we did find partial evidence that L2 learners of english have delayed focus processing for English focus. However, this result is tenuous. While both our replication and \cite{Ge2021} showed evidence of differences in eye-movement patterns during focus processing, our results deviated in almost every possible way. In some ways, we found the oppostite of the original study. That is, Dutch speakers have more competitor focus fixation than English speakers earlier on.  

The current study did not replicate the key findings of \citep{Ge2021}. While our results revealed some statsitically significant differences in eye fixation patterns across object focused and verb focused sentences, the nature of these results did not aligh with \citep{Ge2021}. These discrepancies raise important questions about the role of methodological differences, population-level effects, and statistical interpretation in replication research. 

There are at least three possible reasons for these discrepencies: difference between data collection methods, false positives, and group level differences. As far as difference in data collection method. \citep{Ge2021} conducted their study in a controlled lab setting, using high-precision eye-trackers with different frame rates depending on the participant group. L1 Dutch speakers were tested in the Netherlands using an eye-tracker with a higher sampling rate (500 Hz), whereas L1 English speakers were tested in Hong Kong with a lower sampling rate (300 Hz). This difference in eye-tracking setups means that the timing resolution of fixations was not uniform across groups, potentially influencing the reported processing differences. In contrast, our study utilized web-based eye-tracking, which introduces additional variability due to differences in screen resolution, calibration accuracy, and participant engagement. But even with these differences our participants all used webcams and any variability in fixation capturing was random across groups (low and high frame rates in both groups). The choice to move to web-based eye-tracking here then can be seen as a 双刃剑 (double edged swored). Yes, web-based eye-tracking is lower frame rate and more variable. But the variability is consitent across randomly sampled populations (population is discusssed below). As a result, it is possible that our inability to replicate their findings is at least partially due to the differences in data collection methods rather than differences in cognitive processing.

Secondly, the presence of false positives and false negatives in either our study or \citep{Ge2021} could contribute to discrepancies between our results and theirs. While our sample sizes (Dutch = 31, English = 61) are comparable to those of \citep{Ge2021} (Dutch = 35, English = 40), the number of statistical tests conducted per language (9) increases the likelihood of Type I errors. Given this, there is a 59.34\% probability of obtaining at least one false positive in both studies, making it crucial to interpret significant findings with caution. The risk of false negatives is more complex to quantify. If prosodic effects exist but are small, our sample sizes may be underpowered to detect them, leading to Type II errors. However, estimating this risk is particularly challenging due to the lack of established effect sizes for prosody in focus processing. Since this field is still emerging, future research should aim to establish reliable effect size estimates to improve statistical power calculations and minimize both false positives and false negatives.

Third, although we followed a similar recruitment strategy to \citep{Ge2021}, differences in L2 proficiency, English exposure, or individual cognitive-perceptual abilities between participant samples may have contributed to the discrepancies in findings. Additionally, while \citep{Ge2021} labeled their L1 English group as a control, this group was recruited in Hong Kong, meaning they may not be directly comparable to other L1 English-speaking populations. Furthermore, exposure to or experience with tone languages like Cantonese and Mandarin may have influenced focus processing patterns. The greater reliance on lexical pitch in these languages could have heightened participants’ attentional allocation to prosodic cues in English, potentially enhancing sensitivity to focus marking. However this is pure speculation as linguistic experience of participants in the \citep{Ge2021} was not reported. However, If such differences in prosodic attention played a role in the original findings, they may not generalize across populations with different linguistic backgrounds, helping to explain why our study did not replicate the same effects. This suggests that focus processing may be more variable than previously assumed, and that group-level trends in prosodic cue integration may not be as stable across studies. Further disuccusion of individual difference can be found in our extension.

In sum, while we found statistically significant effects in our data, it is important to emphasize that our fidelity based replication did not test for direct interactions between language groups. Similarly, Ge et al. (2021) conducted separate statistical analyses for each group, meaning that group-level differences were inferred rather than explicitly tested via interaction effects. The fact that our study did not replicate these key findings from \citep{Ge2021} highlights the complexities of replication in psycholinguistics and second language research. This methodological limitation raises concerns about the robustness of previously reported L1-L2 differences, particularly given the potential for false positives when conducting multiple separate statistical tests. If group differences were found in one study but not another, this does not necessarily indicate a true underlying cognitive difference; rather, it could be the result of analytical choices or sample variability. This is particularly important when working with small effect sizes, where statistical significance may not always equate to meaningful or replicable findings. Given these considerations, our fidelity replication highlights the importance of directly testing interactions which we in term do in the next secion.


\subsection{Replication vs. refinement: Balancing rigor and practicality}

A key limitation of both our fidelity replication and the original study by \citep{Ge2021} is the separation of L1 English and L1 Dutch participants into independent statistical models. While this approach may have been necessary in \citep{Ge2021} due to differences in eye-tracking hardware across participant groups, it prevented a direct statistical test of whether group differences were robust or whether they emerged as an artifact of separate analyses. To address this, we refined our analysis by adopting a time-sensitive model that accounts for gradual fixation changes rather than treating time as discrete bins. Using Generalized Additive Models (GAMs), we assessed whether prosodic effects emerged immediately or developed over time. This refined approach allowed us to test whether L1 speakers truly showed an early fixation advantage and whether L2 speakers were uniformly delayed in processing prosodic focus cues.

The refined analysis produced three key findings that deepen our understanding of prosodic cue integration.

First, our results challenge the assumption that L1 English speakers use prosodic focus cues immediately. In \citep{Ge2021}, English speakers were reported to fixate on focus-marked referents earlier than Dutch speakers, suggesting that native speakers efficiently integrate prosodic cues from the outset. However, our target model or competitor model for phrase one did not find a main effect of L1 early in processing. Instead, fixation patterns gradually increased over time, as shown by a main effect of time With object focus leading to greater fixations. This means that fixations to the target did not shift immediately following prosodic cues, but rather developed dynamically, suggesting that previous studies may have overestimated the immediacy of prosodic integration.

Second, rather than revealing a uniform L2 delay, our results show that Dutch speakers were not consistently slower than English speakers in processing prosodic focus cues. In fact, the interaction btween stress and L1 may indicate this difference is negative for both target and consonant indicating less english looks overall for both target and competitor rather than an L1 advantage. This contradicts the expectation that Dutch participants should show a general delay in focus-driven fixations. However, a three-way interaction between time, L1, and stress revealed that L1 English speakers gradually developed a fixation preference for the target in verb-stressed sentences. This suggests that while Dutch speakers did not initially lag behind, L1 English speakers showed a more stable late-stage fixation pattern over the first phrase. In other words, the native speaker advantage may be one of long-term consistency rather than early efficiency.

Third, our refined approach revealed that group-level differences were previously overstated due to statistical modeling choices. In \citep{Ge2021}, L1 and L2 differences were inferred from separate statistical models rather than directly tested as an interaction. Our unified model revealed a negative interaction between L1 and stress in the first phrase competitor model, indicating that English speakers showed fewer fixations to the competitor during verb-stressed sentences. At the same time, a negative interaction between time and stress showed that fixations to the competitor decreased more sharply for English speakers as time progressed. These findings indicate that differences between L1 and L2 participants were not static processing deficits but instead emerged gradually through dynamic changes in fixation patterns over time.

These results have key implications for how we interpret prosodic focus effects in L1 vs. L2 processing. If focus-driven fixations are not immediate, then L2 delays may not reflect an inherent cognitive limitation but rather a difference in cue weighting over time. Similarly, if L1 advantages emerge later in processing rather than earlier, then the assumption that native speakers automatically integrate prosodic cues faster may need to be reconsidered. Finally, the fact that L1-L2 differences were highly sensitive to analytical approach underscores the need for caution when making strong claims about group differences based on discrete time bins.

Unlike our fidelity analysis, our refined analysis does not overturn the key findings of \citep{Ge2021}, but it suggests that their interpretation may have been overly rigid. Rather than a clear and immediate L1 advantage, our results indicate that prosodic cue integration is dynamic, shifting over time rather than appearing as a static group-level effect. By adopting an approach that accounts for these time-dependent changes, we provide a more nuanced understanding of how both native and non-native speakers process prosodic focus cues.

\subsection{Exploratory Extension}




\subsection{Moving Beyond L1-L2 Differences}

In sum, our exploration findings extend our replication and refinement by demonstrating that prosodic focus processing is not solely driven by language background (L1 vs. L2) but is shaped by a combination of acoustic properties of stimuli and listener-specific traits. While \citep{Ge2021} attributed differences in focus processing to L1 effects, our results suggest that listener-specific auditory and cognitive abilities play a more central role than previously assumed.

First, fixations to focus-marked referents were systematically modulated by acoustic properties such as pitch range and word duration, rather than being an invariant effect of prosodic marking. This suggests that prosodic processing is highly stimulus-dependent, meaning that prosodic focus effects are not universal but contingent on specific acoustic realizations of stress and prominence.

Second, L1-L2 differences in prosodic processing were less stable than assumed. While previous studies suggested that non-native listeners exhibit a uniform delay, our results show that fixation patterns are better predicted by individual differences in auditory sensitivity tied to specific acoustic propoerties of the sitmuli \cite{xie2023adaptive}. In other words, prosodic processing is highly individualized, and L2 delays may reflect perceptual and cognitive variability rather than a fixed group-level effect.

Finally, our results highlight the importance of including individual differences in speech perception research L1 and L2 research. By integrating auditory perception, cognitive control, and lexical proficiency, we provide a more nuanced account of prosodic focus processing, moving beyond broad group-level contrasts to uncover how individual listeners dynamically adapt to prosodic cues in real time.


\subsubsection{Limitations}

Studies chosen for replication often fail to clearly delineate between confirmatory analyses that test an explicit hypothesis and exploratory analyses that are post hoc in nature (see \cite{Yanai2020} for discussion of daytime- and nighttime-science). This lack of transparency complicates replication efforts, particularly when determining whether replications should adhere strictly to the original analytical approach, even if it was statistically less rigorous, or refine the methodology in alignment with current best practices, which in turn introduces unintended differences that can affect the comparability between the replication and the original study \cite{mcmanus2022replication}.

