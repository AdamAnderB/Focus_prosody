Replication studies often blur the line between confirmatory and exploratory analyses, complicating efforts to distinguish between fidelity and refinement (see \cite{Yanai2020} on daytime- and nighttime-science). This lack of transparency forces a tradeoff—should replications adhere to the original, even if statistically flawed, or refine the approach at the risk of introducing unintended differences \cite{mcmanus2022replication}? Our FiREE framework addresses this challenge, demonstrating that prosodic focus processing is more variable than previously assumed and that strict replication alone may not capture the full picture. Rather than a binary success-or-failure, replication should be seen as an iterative process—one that strengthens methodological rigor while uncovering new theoretical insights.