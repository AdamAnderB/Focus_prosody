\section{Conclusion}
Replication studies often blur the line between confirmatory and exploratory analyses, complicating efforts to distinguish between fidelity and refinement \parencite{Yanai2020}. This lack of transparency raises an important question: should replications adhere to the original analyses, even if statistically flawed, or refine the analyses at the risk of introducing unintended differences \cite{mcmanus2022replication}? Our FiREE framework addresses this challenge by stressing fidelity and refinement while also encouraging exploratory extensions into theoretically relevant areas. Our FiREE replication demonstrated that prosodic focus processing is more variable than previously assumed and that strict replication alone may not capture the full picture. Rather than a binary success-or-failure, replications should be seen as an iterative process—one that strengthens methodological rigor while uncovering new theoretical insights.